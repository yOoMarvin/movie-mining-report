 % do not change these two lines (this is a hard requirement
% there is one exception: you might replace oneside by twoside in case you deliver 
% the printed version in the accordant format
\documentclass[11pt,titlepage,oneside,openany]{book}
\usepackage{times}

%personally added
\usepackage{color}

\usepackage{graphicx}
\usepackage{latexsym}
\usepackage{amsmath}
\usepackage{amssymb}


\usepackage{ntheorem}

% \usepackage{paralist}
\usepackage{tabularx}

% this packaes are useful for nice algorithms
\usepackage{algorithm}
\usepackage{algorithmic}

% well, when your work is concerned with definitions, proposition and so on, we suggest this
% feel free to add Corrolary, Theorem or whatever you need
\newtheorem{definition}{Definition}
\newtheorem{proposition}{Proposition}


% its always useful to have some shortcuts (some are specific for algorithms
% if you do not like your formating you can change it here (instead of scanning through the whole text)
\renewcommand{\algorithmiccomment}[1]{\ensuremath{\rhd} \textit{#1}}
\def\MYCALL#1#2{{\small\textsc{#1}}(\textup{#2})}
\def\MYSET#1{\scshape{#1}}
\def\MYAND{\textbf{ and }}
\def\MYOR{\textbf{ or }}
\def\MYNOT{\textbf{ not }}
\def\MYTHROW{\textbf{ throw }}
\def\MYBREAK{\textbf{break }}
\def\MYEXCEPT#1{\scshape{#1}}
\def\MYTO{\textbf{ to }}
\def\MYNIL{\textsc{Nil}}
\def\MYUNKNOWN{ unknown }
% simple stuff (not all of this is used in this examples thesis
\def\INT{{\mathcal I}} % interpretation
\def\ONT{{\mathcal O}} % ontology
\def\SEM{{\mathcal S}} % alignment semantic
\def\ALI{{\mathcal A}} % alignment
\def\USE{{\mathcal U}} % set of unsatisfiable entities
\def\CON{{\mathcal C}} % conflict set
\def\DIA{\Delta} % diagnosis
% mups and mips
\def\MUP{{\mathcal M}} % ontology
\def\MIP{{\mathcal M}} % ontology
% distributed and local entities
\newcommand{\cc}[2]{\mathit{#1}\hspace{-1pt} \# \hspace{-1pt} \mathit{#2}}
\newcommand{\cx}[1]{\mathit{#1}}
% complex stuff
\def\MER#1#2#3#4{#1 \cup_{#3}^{#2} #4} % merged ontology
\def\MUPALL#1#2#3#4#5{\textit{MUPS}_{#1}\left(#2, #3, #4, #5\right)} % the set of all mups for some concept
\def\MIPALL#1#2{\textit{MIPS}_{#1}\left(#2\right)} % the set of all mips


% custom stuff
\usepackage[hidelinks]{hyperref}


\begin{document}

\pagenumbering{roman}
% lets go for the title page, something like this should be okay
\begin{titlepage}
	\vspace*{2cm}
  \begin{center}
   {\huge Mining the Success for Movies \\}
   \vspace{2cm} 
   {\Large Student Project Data Mining HWS17\\
   Team 6\\}
   \vspace{2cm}
   {\Large Presented by \\}
   \vspace{0.5cm}
    {Steffen Jung \\
    Adrian Kochsiek \\
    Martin Koller \\
    Marvin Messenzehl \\
    Daniel Szymkowiak \\
   }
   \vspace{1cm} 
   { Submitted to the\\
    Data and Web Science Group\\
    Prof.\ Dr.\ Heiko Paulheim\\
    University of Mannheim\\} \vspace{2cm}
   {October - December 2017}
  \end{center}
\end{titlepage} 

% no lets make some add some table of contents
\tableofcontents
\newpage

%\listofalgorithms

%\listoffigures

%\listoftables

% evntuelly you might add something like this
% \listtheorems{definition}
% \listtheorems{proposition}

\newpage


% okay, start new numbering ... here is where it really starts
\pagenumbering{arabic}

%%%%%%%%%%%%%%%%%%%%%%%%%%%%%%%%%%%%%%%%

% INPUTS
\chapter{Application Area and Goals}
\label{cha:area_goals}
This paper represents a documentation for the data mining project \textit{"Mining the Success for Movies"}\footnote{ Information in this paper refers to the (sample) dataset and python scripts handed in for a classification problem}. The structure of this paper follows the classical data mining process. Chapter \ref{cha:area_goals} provides an overview of the problem the project is based on and is complemented by the goals and objectives of this project. Afterwards, chapter \ref{cha:data_selection} deals with the structure and size of the data. Here, a closer look will be taken at the dataset at hand. Questions that had to be answered were for example which information were provided in the original dataset or which problems were identified concerning for instance outliers or missing values. Upon that, chapter \ref{cha:preprocessing_transformation} explains which preprocessing steps had to be taken in order to cleanse the dataset to prepare it for the data mining step and model learning. Chapter \ref{cha:data_mining} describes which data mining techniques regarding algorithms and parameters were used to learn an expedient model in respect of the goals set in Chapter \ref{cha:area_goals}. Finally, chapter \ref{cha:interpretation_evaluation} closes this paper by describing which insights could be won for the problem at hand. Here, a critical reflection is delineated how the model could be improved further in order to provide even more precise results.

\section{Problem Statement}
Already well before new movies are being produced, every stakeholder certainly is interested in the monetary success of the given movie. In order to predict the success costly methods are being applied, such as market investigations or analyses.

The benefit which Data Mining brings to the analysis of large datasets, can also be transferred onto the stated problem of predicting a movie's success. Based on given data of already released movies and successful movies in the past, a model is being learned which shall be applied on upcoming or planned movies. In order to learn and apply the model various pieces of information are taken into account. Just a few are budget, revenues, runtime, genre and information on the release. Information on the dataset and on all preprocessing methods which were applied will be provided in chapter \ref{cha:data_selection}.

\section{Goals}
The goal of this project is to learn a model which will predict how successful a not yet released movie will be. This is done by using common data mining techniques in the Python programming language. As a main objective the question \textit{"Based on revenue, will the movie be popular or will it be a flop?"} shall be answered for all possible combinations of information on a new movie as precisely as possible.
In order to be as precisely as possible, not only different algorithms are being tested, but also parameter tuning is being applied with different performance measures \footnote{Further information on applied techniques and evaluation methods is provided in chapter \ref{cha:data_mining}}.









\chapter{Data Selection and Preprocessing}
\label{cha:data_selection}

\section{Structure of data and data exploration}
\label{sec:data_exploration}
The selected dataset onto which a classification model shall be learned is provided by Kaggle\footnote{2017 Kaggle Inc}. It is named \textit{The Movies Dataset}\footnote{Link to the dataset: \hyperref[https://www.kaggle.com/rounakbanik/the-movies-dataset]{https://www.kaggle.com/rounakbanik/the-movies-dataset}} and contains metadata of approximately 45,000 movies in its raw format. It is provided and updated by Rounak Banik. The complete dataset consists of several files in \textit{csv}-format containing data about movie casts, metadata and external scores. The main file used during preprocessing is named \textit{movies-metadata.csv}. This \textit{csv}-file contains 24 columns, which can be seen in the graphic below. In addition to that, information from the file \textit{credits.csv} is also extracted. It is described in section \ref{cha:preprocessing}.
\begin{figure}[ht]
	\centering
		\includegraphics[width=\textwidth]{images/Raw_dataset_headers.png}
	\caption{Columns of the file \textit{movies-metadata.csv}}
\end{figure}

%Due to the multitude of attributes, it is essential to extract the relevant values for the implementation of a well-functioning classifier. Many values have no effect on the financial success of a movie and can therefore be ignored. This includes, for example, the columns...

The structure of the individual columns is very different. In addition to boolean values, strings and numeric float values (e.g \textit{budget} or \textit{runtime}), many attributes contain longer texts (e.g \textit{overview}), arrays or even a list of JSON objects (e.g. \textit{production\_countries}). The different formats must be taken into account in the preprocessing and need to be processed and extracted individually.
%\\\\
%\section{Basic data exploration}

The overall quality of the dataset varies significantly. This can be due to the fact that it is maintained by a community and is not provided by a larger organization or company. Therefore, a demanding quality assurance process is difficult to realize. It is an important quality aspect of a dataset that it contains values for the majority of its records. The examined dataset has a high amount of records containing missing values. This is especially the case for older movies (before 1960), which contain only few information. Critical for this project is the availability of the attribute values \textit{revenue} and \textit{budget}, which are used to determine the financial success of the movie. Here around 34,000 records are containing zero or missing values in either the \textit{revenue} or the \textit{budget} column.

In addition to that, there is no indication about the currencies of numerical financial attributes. Furthermore, the values of these attributes are inconsistent in respect to scaling. For example a movie having an actual revenue of \$ 312MM and a budget of \$ 240MM is given with a \textit{revenue} value of \textit{240,000,000} as a float number whereas the \textit{budget} value contains the float number \textit{312}. Another source of inconsistency is discovered after randomly cross-checking the values contained in the dataset with available box office numbers online. These checks indicate that different market assumptions (e.g. US market only vs. worldwide) were taken into account for the values contained in the dataset. For the successful outcome of this project it is crucial to keep those aspects in mind for later preprocessing steps, otherwise the results will be distorted.

Speaking of those issues it is also worth mentioning that the chosen dataset contains movie metadata from the past 60 years. With respect to today's economics a lot of things changed in movie production and consumption. Aspects that have influenced movie economics are for example price structures, inflation and globalization. Naturally, also consumer behavior changed drastically. Therefore, it was decided to introduce a new attribute to the dataset called \textit{productivity}. 
\begin{wraptable}{l}{0.55\textwidth}
	\begin{tabular}{| l | l |}
	\hline
	Average budget & \$ 31,662,585 \\ \hline
	Average revenue & \$ 92,059,210 \\ \hline
	Average productivity & 2.8 \\ \hline
	Most common genres & Drama, Comedy,\\ & Thriller \\ \hline
	Average runtime & 94 min. \\ \hline
	\end{tabular}
	\caption{Basic insights from the dataset}
	\label{tbl:insights}
\end{wraptable} 
The \textit{productivity} describes the ratio of \textit{revenue} and \textit{budget} of a movie and expresses its success in this project. A \textit{productivity} of $1.0$ implies that the budget to produce the movie was covered and anything above $1.0$ is assumed as profit.
%\\
In order to get a better understanding of the data and its relations, table \ref{tbl:insights} presents some average numbers of crucial columns and the most common genres.

\chapter{Preprocessing and Transformation}
\label{cha:preprocessing_transformation}

In order to pick out columns that have a significant impact on forecasting a movie's success, the following assumptions concerning the importance of information in the metadata were made:
\begin{itemize}
	\item The budget and revenue are crucial numbers.
	\item The release year has an impact on numbers such as budget and revenue.
	\item The genre, the production company and the runtime are important information.
\end{itemize}

In order to transform the data into a suitable representation for forecasting a movie's success, preprocessing was mandatory. All preprocessing steps including either the following: Merging of columns, binning of features, extracting information out of columns, one hot encoding or normalizing. Before those steps could be applied, some columns were dropped out.

Display which ones here!

%can be considered to be in one of three different categories: Dropping columns, adding features and extracting and encoding information.

\begin{figure}
\includegraphics[width=\textwidth]{images/3_features.png}
\caption{Features created during preprocessing}
\label{img:features}
\end{figure}


\begin{itemize}
	\item Transform data into a representation that is suitable for the chosen data mining methods
	\begin{itemize}
		\item number of dimensions
		\item scales of attributes (nominal, ordinal, numeric)
		\item amount of data (determines hardware requirements)
	\end{itemize}
	\item Methods
	\begin{itemize}
		\item Aggregation, sampling
		\item Dimensionality reduction / feature subset selection
		\item Attribute transformation / text to term vector
		\item Discretization and binarization
	\end{itemize}
	\item Good data preparation is key to producing valid and reliable models
	\item Data preparation estimated to take 70-80\% of the time and effort of a data mining project!
\end{itemize}


\section{Preprocessing steps according to Python script}

\section{A list of problems we encountered}
\begin{enumerate}
	\item \textbf{list further problems we had and solved!}
	\item Prod. Comp.: Same prod. company named differently -> using Regex to solve (Steffen)
	\item dataset: 5 datasets have duplicates
\end{enumerate}

\chapter{Data Mining}
\label{cha:data_mining}
\begin{itemize}
	\item Input: Preprocessed Data
	\item Output: Model / Patterns
\end{itemize}

\begin{enumerate}
	\item Apply data mining method
	\item Evaluate resulting model / patterns (using P, R, F1, not accuracy)
	\item Iterate:
	\begin{itemize}
		\item Experiment with different parameter settings
		\item Experiment with different alternative methods – Improve preprocessing and feature generation – Combine different methods
	\end{itemize}
\end{enumerate}

\section{Algorithms}
\begin{itemize}
	\item \textbf{Random Forest}
	\item \textbf{Decision Tree}
	\item KNN
	\item Bayes
	\item NeuralNet
	\item svc
\end{itemize}

\section{Three best performing algorithms}
\begin{itemize}
	\item Pick best three algos
	\item GridSearch
	\item Why does each classifier perform how it performs (unausgeglichene Klassen, ...)?
\end{itemize}

\chapter{Interpretation / Evaluation}
\label{cha:interpretation_evaluation}
Maybe also prospect?
\begin{itemize}
	\item Output of Data Mining
	\begin{itemize}
		\item Patterns
		\item Models
	\end{itemize}
	\item In the end, we want to derive value from that, e.g.,
	\begin{itemize}
		\item gain knowledge
		\item make better decisions
		\item increase revenue
	\end{itemize}
\end{itemize}
\input{content/chapter_6_prospect.tex}

%\input{./content/outline.tex}

%\bibliographystyle{plain}
%\bibliography{literature}




%not needed as stated in dws_vorlage
%\appendix

%\chapter{Program Code / Resources}
%\label{cha:appendix-a}

%The source code, a documentation, some usage examples, and additional test results are available at ...

%They as well as a PDF version of this thesis is also contained on the CD-ROM attached to this thesis.

%\chapter{Further Experimental Results}
%\label{cha:appendix-b}

%In the following further experimental results are ...




\newpage


\pagestyle{empty}


%\section*{Ehrenw\"ortliche Erkl\"arung}
%Ich versichere, dass ich die beiliegende Master-/Bachelorarbeit ohne Hilfe Dritter
%und ohne Benutzung anderer als der angegebenen Quellen und Hilfsmittel
%angefertigt und die den benutzten Quellen w\"ortlich oder inhaltlich
%entnommenen Stellen als solche kenntlich gemacht habe. Diese Arbeit
%hat in gleicher oder \"ahnlicher Form noch keiner Pr\"ufungsbeh\"orde
%vorgelegen. Ich bin mir bewusst, dass eine falsche Er- kl\"arung rechtliche Folgen haben
%wird.
%\\
%\\

%\noindent
%Mannheim, den 31.08.2014 \hspace{4cm} Unterschrift

\end{document}
