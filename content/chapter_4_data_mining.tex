\chapter{Data Mining}
\label{cha:data_mining}

After preprocessing the data, the next step is to build a classifier to predict the success of a movie. To find the best classifier a set of different algorithms is evaluated:
K-Nearest Neighbor, 
Naive Bayes, 
Support Vector Classifier, 
Neural Net, 
Decision Tree and 
Random Forest.

The first goal of the analysis was to predict the success in five different classes. Since an analysis in that detail with the given data set is very unprecise, a binary classifier was created.
For all of the named algorithms an individual feature selection is applied to get the best results. The important features are selected by a greedy feature wrapper. This feature wrapper starts with the full set of features and drops features as long as the result improves. The results are checked with a ten fold cross-validation. Additionally to the selection of features, filters for features like the number of actors and number of production companies can be applied, to only use a certain percentage of the most occurring of them. \\
In the next step some hyperparameter tuning is needed to improve the classifiers. To find the best parameter setting, a gridsearch in combination with a ten-fold cross-validation, scoring the highest F1-score (macro), is applied to the classifiers.
The achieved F1-scores of the three best binary classifier are listed in table \ref{tab:binary_classifier}. 
At first the classifiers are scoring the micro F1-score. Even though the result looks promising at the beginning, the classifier are mostly guessing the majority class in this setting. For that reason every classifier is scoring on the macro F1-score. Since the data set for the binary classifier is unbalanced with a ratio of 76\% "successful" and 24\% "unsuccessful", the next step to improve the classifiers is to balance the train set. That for the stratified cross validation is used to balance the train set. For further improvement the data set can be downbalanced by shrinking the majority class of the train set to the size of the minority class. But since the data set is quite small already after the preprocessing, the downsampling worsens the classifiers. To not shrink the dataset any further upsampling can be used to increase the minority class of the train set to the size of the majority class. In the following section the decision tree, random forest and the support vector classifier are examined in more detail.

\begin{center}
\begin{table}
	\begin{tabular}{ | p{3.5cm} | p{1.5cm} | p{1.5cm} | p{2cm} | p{2cm} |}
    \hline
    Algorithm & F1 Macro & F1 Micro & Downsampled Macro  & Downsampled Micro\\ \hline
    Decision Tree & 56.5\% & 75.5\% & 61.9\% & 63.0\% \\ \hline
    Random Forest & 58.7\% & 76.0\% & 80.6\% & 81.5\% \\ \hline
    Support Vector Classifier & 56.5\% & 75.0\% & 60.7\% & 60.8\% \\
    \hline
    \end{tabular}
    \caption{Binary Classifier Results} 
    \label{tab:binary_classifier}
\end{table}
\end{center}
\section {Classifiers}
\paragraph{Decision Tree}
One way to deal with the issue of an unbalanced data set is to use a tree-structured classifier, since the hierarchical structure allows them to learn signals from both classes.
To find the best parameter setting for the decision tree the grid-search tests the parameters split criterion, max depth, minimum samples to split and the class weight in a ten fold cross-validation. After dropping the "quarter", "runtime" and "adult" columns, the classifier scores without any sampling, no class weight, an entropy split criterion and a max depth of 100 an F1 score of 58\%. Both the down- and the upsampling worsens the classifier.

\paragraph{Random Forest}
Like the decision tree, random forest builds  a hierarchical structure, which can handle unbalanced data sets better than other algorithms. In addition it corrects the overfitting habit of a decision tree. In the gridsearch hyperparameters like the split-criterion, number of features, the minimum sample to split and whether bootstrap samples are used or not are evaluated. After dropping the actors and the release quarter, the classifier scores a F1 score of 59\%. As with the decision tree neither down- nor upsampling the train set improves the result.

\paragraph{Support Vector Classifier}
\section{Evaluation and Prospect}
\paragraph{Evaluation}
As discovered in the previous paragraphs, some classifiers perform better than others given the provided task. Big impact is due to the amount of data which each classifier receives. From 45000 in the beginning a reasonable amount of data needed to be dropped in order to maintain quality. The final dataset of about ~3800 rows can be considered as small. 

Decision trees seem to be best performing in this specific task, since they not rely on signals from both classes. The same concept is performed by the Random forrest classifier. This classifier extends the benefit from a decision tree, and evaluates it with a forrest out of trees. Therefore also large datasets can easily be evaluated and learned. Since each decision tree is build and classified separately high performance is guaranteed.

Other classifiers as for example Naive Bayes perform not as well as the trees. One possible reason could be the assumptions that no correlation between the features should exist. However, in this specific task movies with more famous actors usually generate a bigger budget. Still a reasonable good score can be reached, which might be due to the cause that only a small interdepence is influencing the tree.


%\begin{itemize}
%	\item very small dataset -> not all algorithms perfom very well
%	\item Example numbers -> some perform better: why? 
%	\item Analysis of values
%\end{itemize}

