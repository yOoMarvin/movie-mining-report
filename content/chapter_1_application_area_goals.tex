%\chapter{Application Area and Goals}\label{cha:area_goals}
\begingroup
\renewcommand{\cleardoublepage}{}
\renewcommand{\clearpage}{}
\chapter{Application Area and Goals}\label{chap:area_goals}
\endgroup


This paper represents a documentation of the data mining project \textit{"Mining the Success for Movies"}\footnote{Information in this paper refers to the (sample) dataset and python scripts handed in for a classification problem. The repository can be found at \hyperref{https://github.com/yOoMarvin/movie-mining}{external}{github}{on GitHub under: https://github.com/yOoMarvin/movie-mining}}. The structure of this paper follows the classical data mining process: The data selection and preprocessing are covered in chapter \ref{cha:data_selection} and the data mining with an evaluation of results and a prospect are covered in chapter \ref{cha:data_mining}.

%Chapter \ref{cha:area_goals} provides an overview of the problem the project is based on and is complemented by the goals and objectives of this project. Afterwards, chapter \ref{cha:data_selection} deals with the structure and size of the data. Here, a closer look will be taken at the dataset at hand. %Questions that had to be answered were for example which information was provided in the original dataset or which problems were identified concerning for instance outliers or missing values. 
%Upon that, chapter \ref{cha:preprocessing} explains which preprocessing steps had to be taken in order to cleanse the dataset to prepare it for the data mining step and model learning introduced in chapter \ref{cha:data_mining}. Here, the used data mining techniques are described regarding algorithms and parameters that were used to learn an expedient models in respect of the goals set in Chapter \ref{cha:area_goals}. Additionally, gained insights of the problem at hand are presented and a critical reflection is delineated how the model could be improved further in order to provide even more precise results.

\paragraph{Problem Statement}
Before new movies are being produced, every stakeholder is interested in the monetary success of the intended movie. In order to predict the success, costly methods are being applied, such as market investigations or analyses. The benefit of Data Mining to the analysis of large datasets can also be transferred to the stated problem of predicting a movie's success. %In order to learn and apply the model, various pieces of information are taken into account. Just a few are budget, revenues, runtime, genre and information on the release. Information on the dataset and on all preprocessing methods which were applied will be provided in chapter \ref{cha:data_selection}.

\paragraph{Goals}
The goal of this project is to learn a model which will predict how successful a not yet released movie will be. This is done by using common data mining techniques in the Python programming language. Several packages were used for different tasks. For example, the packages \textit{pandas} and \textit{numpy} were used for working with dataframes and numerical values. For creating graphics and plots, the chosen package was \textit{matplotlib}. In addition to that, parts of the  Python machine learning library \textit{scikit learn} were used for building classification models and evaluating them. As the main objective the question \textit{"Based on revenue, will the movie be popular or will it be a flop?"} shall be answered for all possible combinations of information on a new movie as precisely as possible.
%In order to be as precise as possible, not only different algorithms are being tested, but also parameter tuning is being applied with different performance measures \footnote{Further information on applied techniques and evaluation methods is provided in chapter \ref{cha:data_mining}}.








 