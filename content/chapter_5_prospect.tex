\paragraph{Prospect}
\label{cha:prospect}
In order to extend this project and to improve the performance of the models, various approaches can be considered for future research. 

One of the crucial challenges regarding the missing values of budget and revenue could be targeted with additional information sources. Where this project relies on the external sources from \textit{IMDd} and \textit{The Numbers} also other sources can be taken into account. The more reliable information provided, the easier the dataset can be expanded back to the original set of 45,000 entries.

\normalsize Including more powerful features might be another possibility to increase the value of the dataset. In this project, members of the crew are limited to actors and directors of each movie. Improvement might be reached by adding more famous and known members, e.g. writers. Additionally, the popularity of cast and crew members, particularly actors, directors and writer, should be taken into account. By this a more fine-grained distinction of importance in between the same class of members could be achieved.

Another important indicator of the quality of a movie is its content. In order to take this into account, a keyphrase extraction of the synopsis of each movie could derive additional features to distinguish between the movies. Further research should therefore focus on bringing all these information together and by that improve the data mining process.

%\begin{itemize}
%	\item find other data sources for revenue / budget values to expand the data set with additional movies
%	\item try other feature selection methods (backward wrapper, filter)
%	\item add additional features
%	\begin{itemize}
%		\item popularity of the actors / crew
%		\item add additional crew members (e.g. writer)
%		\item extract key phrases from the synopsis of the movie
%	\end{itemize}
%\end{itemize}
%
